\documentclass[12pt,a4paper]{article}
\usepackage{times}
\usepackage{durhampaper}
\usepackage{harvard}

\citationmode{abbr}
\bibliographystyle{agsm}

\title{User Interaction Discovery in Virtual Environments}
\student{L. A. Sutton}
\supervisor{W. Li}
\degree{BSc. Natural Sciences}

\date{}

\begin{document}

\maketitle

\begin{abstract}

{\bf Context/Background}

{\bf Aims}

{\bf Method}

{\bf Results}

{\bf Conclusions}

\end{abstract}

\begin{keywords}
user interaction; virtual environments; visualisation; clustering
\end{keywords}

\section{Introduction}

In the 21st century, people spend more time than ever interacting in virtual environments, whether that takes the form of social networking, email or video games. Many previous attempts have been made to visualise the structures that form within these environments \cite{freeman2000visualizing}

In this project I have taken 'User Interaction' to mean any way in which users consciously affect other users so that the other users would be able to identify the specific other users they were interacting with. This could take place over a period of time of be instantaneous; it could be between two users or many; it could be a single event or it could be ongoing. For example I will consider interactions such as users sending emails between one another but I will not consider interactions such as a user's advertising preference changing what another user sees as this falls beyond the scope I have layed out.

Virtual environments in this project will mean any environment in which users are able to interact in the ways I have previously mentioned, mediated by computers. This could be for example a video game, a social networking website or a messaging system such as email.

\subsection{Project Motivation}

While there have been many previous attempts to visualise the structure of virtual environments, these have almost exclusively focused on static representations of the relationships between users. From this a lot of data has been gathered about information presentation.

As has been seen, much of the previous efforts at visualisation and presentation had focused on the static representation of relationships between people in virtual environments. I wanted to extend this in two ways.

Firstly I wanted to extend the existing work that covered these static representations of relationships and take what was learned to apply to dynamic systems of interactions that were able to evolve over time.

Secondly I wanted to move the focus of the systems away from the current system of representing only the relationships between users that are inferred from the interactions. I would instead represent the interactions themselves.

\subsection{Project Aims}

Put aims here

\section{Related Work}

\subsection{Social Network Visualisation}
Network visualisation is already a science with a long history, especially since being able to use computers to position and draw the output. There are currently low-level tools that exist with the intention that they have the necessary flexibility to accommodate a wide variety of visualisation styles and techniques \cite{heer2005prefuse}. As well as this, there are tools that exist to provide numerical analysis \cite{borgatti2002ucinet}. However, these tools have focused on the analysis and visualisation of snapshots of data remaining static, rather than data sets that evolve over time. These have also previously been used to build ways of visualising social network data from Friendster \cite{heer2005vizster} in order to facilitate discovery of more information than would be apparent from other ways of looking the data, as I hope to.

\subsection{Information Presentation}
There is also a wide variety of information on the presentation of data on computer screens. One particularly popular model can be summed up as 'Overview, Zoom, Filter' \cite{shneiderman1996eyes}. In this it is suggested that the initial view of the data should be a movable field of view with emphasis on allowing the user to gain an 'overview' of relevant data and identify areas which will be of interest. Specific areas of interest can then be zoomed in on preserving the context of the overall picture before extra information of areas of interest can be viewed possibly by clicking on them. This paper also talks about things such as the importance of smooth display updates and responsiveness to user input. This is built upon by the ideas of making information more clear by distorting the 'presentation space' \cite{carpendale2001framework}. This is the method used in Vizster and can be seen in common usage in many different data visualisation applications. It imagines that the virtual space in which the data is presented is a real material that can be stretched and viewed through a movable lens as necessary to make the relevant areas of the information more clear. These ideas area also expanded on further to see what kinds of lenses are suitable for which purposes, and suggests a mathematical framework for implementing such a lens \cite{leung1994review}. Contained in all of these articles on visualisation are also many suggestions for evaluation of data presentation on computers for example by the ability to maintain context between switching between the three areas on the 'Overview, Zoom, Filter' model and the responsiveness to user input that is possible.

\subsection{Graph Drawing}
There is also previous literature on the drawing of graphs in aesthetically pleasing ways. Almost all current research makes use of a force directed spring layout. In this algorithm, each node is modelled is repelling each other node and the edges between them are modelled as springs\cite{fruchterman1991graph}. Included in these papers are suggestions for the strength of the attractive and repulsive forces different distances and the size of graph that this can be expected to create. However, this algorithm doesn't scale well with rapidly increasing numbers of vertices. It has been pointed out that with a large number of nodes, calculating the layout in this way is very expensive in terms of computing power. However, with the correct optimisations it is possible to reduce the complexity to $o(nlog(n))$ \cite{barnes1986hierarchical}.

\subsection{Modelling Social Networks}
The ability to produce networks of relations and interactions from many different data sources is also explored in a variety of different papers. For example, networks of social interaction have been produced from a history of email correspondence within an organisation \cite{fisher2004social}. Here other relevant ideas are explored such as the privacy implications of collecting data on a large scale and the ability to reconstruct the whole graph from only partial data. The same has also been achieved using the transcript of an internet relay chat \cite{mutton2004inferring} again struggling with the problem of reconstructing a complete graph from partial data. It is then further shown that the same method including the temporal decay of relationships can be applied to other sources of relationship information involving over time such as the plays of William Shakespeare.

\subsection{Categorisation of Interactions}
Previous reserach has also explored categorisation of interactions by their characteristics. This has mainly in the past been applied to social iterations writing 3D virtual environments in which people interact as virtual avatars, referred to as Networked-Virtual Environments. One particular application of this is games \cite{manninen2000interaction}. Here we can see that there is more than one way of categorising interactions, one way being based on their purpose. These papers also show how it is possible for many different modes of interaction to happen simultaneously. It is possible to use communicative action theory to categorise interactions by their purpose, this is extended in other papers by comparing interactions in a selection of game environments \cite{becker2002social}. Extending this to other environments such as the social network, other papers show how much of the interaction that goes on writhing a virtual environment is hidden from the user. We can see just how much data website such as Facebook collect about us including in our making interactions which we wouldn't normally consider meaningful \cite{schneier2010taxonomy}

\subsection{Evolution of social networks}
Ideas of the behaviour of users in social networks have been the subject of many different papers. This includes homophily \cite{adamic2003social} which is the idea that people on social networks tend to associate with people who are similar to themselves in terms of age, political views etc. Work has also been completed on the behaviours of users within a social network and the ways in which interactions can spread behaviour across networks of people represented as graphs. It has been suggested that this can be explained using a virus like model \cite{centola2010spread} in which users pass between susceptible, infected and recovered states, analogous to a computer-virus or a real virus.

\subsection{Detection of clustering}
Detecting features of social networks that are not immediately apparent is also extremely important. We can see that algorithms have been developed that aim to detect communities, related to clustered sections of graph representations of these networks \cite{newman2004fast}. These algorithms can be applied to real world networks with a good degree of success reported in identifying the same communities that the users themselves identify with.

\section{Solution}

\subsection{Graph Layout}

\section{Results}

\section{Evaluation}

\section{Conclusions}

\bibliography{projectpaper}


\end{document}